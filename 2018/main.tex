\documentclass{article}
\usepackage{amsmath,amssymb,amsthm,amsbsy,amsfonts,mathtools}
\usepackage{physics}
\usepackage{url}
\usepackage{hyperref}
\usepackage{tikz}
\usepackage{pgfplots}

\usepackage{listings}
\lstdefinelanguage{julia}{
  basicstyle=\small\ttfamily,
  showspaces=false,
  showstringspaces=false,
  keywordstyle={\textbf},
  morekeywords={if,else,elseif,while,for,begin,end,quote,try,catch,return,local,abstract,function,generated,macro,ccall,finally,typealias,break,continue,type,global,module,using,import,export,const,let,bitstype,do,in,baremodule,importall,immutable},
  escapeinside={~}{~},
  morecomment=[l]{\#},
  commentstyle={},
  morestring=[b]",
}
\lstset{language=julia, numbers=left, numberstyle=\tiny, mathescape=true}

\bibliographystyle{siam}

\usepackage{todonotes}

\author{Yingbo Ma}
\date{\today}
\title{GSoC proposal}

\begin{document}
\maketitle
\tableofcontents

\begin{abstract}
My abstract.
\end{abstract}

\section{The Project}
% TODO: What do you want to have completed by the end of the program?

% TODO: Who’s interested in the work, and how will it benefit them?

% TODO: What are the potential hurdles you might encounter, and how can you
% resolve them?

% TODO: How will you prioritize different aspects of the project like features,
% API usability, documentation, and robustness?

% TODO: What milestones can you target throughout the period?

% TODO: Are there any stretch goals you can make if the main project goes
% smoothly? Tell us how you’re going to wow us with the final result!

\section{Code Portfolio}

\section{Deliverables}
% TODO: Stiffness Detection and Automatic Switching Algorithms

% TODO: SciCompDSL.jl

\section{About Me}
% TODO: What academic, professional or hobby programming experience do you
%have, and how will it help you with your project?

% TODO: Have you contributed to open source projects before? (Link us to some
% issues and patches, if any)

% TODO: Why are you interested in Julia? Have you used it much before? You need
% to demonstrate your ability to use Julia by the beginning of the GSoC. GSoC
% is not for learning the Julia language, though extensive prior experience is
% not required.

% TODO: Do you have the mathematical/scientific background for your project?
% Many of the Julialang projects have a significant portion that require
% technical expertise and applicants need to demonstrate their ability to
% handle the chosen project.

% TODO: How should we contact you? Let us know your email address and GitHub
% username.

\section{Logistics}
% TODO: What other time commitments, such as summer courses, other jobs, planned
%vacations, etc., will you have over the summer?

\bibliography{cite}
\end{document}
